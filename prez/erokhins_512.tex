%\documentclass[dvips, intlimits, 9pt, unicode]{beamer} %для tex -> dvi -> ps -> pdf
\documentclass[pdf, intlimits, 9pt, unicode]{beamer} %Для Latex2Pdf  tex -> pdf
%dvips нужно использовать _только_ если использовать построение слайдов через PostScript
%intlimits - стиль для пределов интегралов (по желанию)
%unicode - обязательно

%Пакеты для русского языка
\usepackage[T2A]{fontenc}
\usepackage[utf8]{inputenc}
\usepackage[english,russian]{babel}

%Пакет для вставки рисунков
\usepackage{graphicx}
%AMS TEX значки и пр.
\usepackage{amssymb}
\usepackage{amsthm}

%разные пакеты
%\usepackage[mathscr]{eucal}
%\usepackage{cite}
%\usepackage{dsfont}
%\usepackage{indentfirst}

%Привычный шрифт для математических формул
\usefonttheme[onlymath]{serif}

%Нужно включать, если используется "тема" (стиль оформления) по умолчанию
\usepackage{beamerthemesplit}

%Общий стиль ("тема") оформления слайдов
%Можно выбрать любую тему в \localtexmf\tex\latex\beamer\themes\theme\
%и ее имя подставить в качестве аргумента в \usetheme
%Требование: черные буквы на белом фоне
\usetheme{Warsaw}

%Более крупный шрифт для подзаголовков титульного листа
\setbeamerfont{institute}{size=\normalsize}

%Задание команды (\bluetext) для выделения конкретным (синим) цветом
%(используйте \alert для выделения цветом выбранной "темы")
\setbeamercolor{bluetext_color}{fg=blue}
\newcommand{\bluetext}[1]{{\usebeamercolor[fg]{bluetext_color}#1}}

%Если используется последовательное появление пунктов списков на слайде
%(не злоупотребляйте в слайдах для защиты дипломной работы), чтобы
%еще непоявившиеся пункты были все-таки немножко видны.
\setbeamercovered{transparent}

\title{Рост двудольных графов с рёберным приоритетом}
\author{Ерохин Станислав Евгеньевич, гр. 512}
\institute{Санкт-Петербургский государственный университет \\
    Математико-механический факультет \\
    Кафедра теории вероятностей и математической статистики \\
    \vspace{0.7cm}
    Научный руководитель:  к.ф.-м.н., Якубович Ю.\,В. \\
    Рецензент: к.ф.-м.н., д. Валландер С.\,С. \\
    \vspace{0.7cm}
}
\date{
    Санкт-Петербург\\
    2015г.
}

\begin{document}

\begin{frame}
    \titlepage
\end{frame}

\begin{frame}
    \frametitle{Содержание работы}
    Дипломная работа состоит из трех частей:

    \begin{itemize}
    %<1->, <2-> после \item означает, что пункты списка будут появляться последовательно, по нажатию клавиши
    %При защите диплома лучше обойтись без этого, так как есть довольно большой шанс, что
    %не будет пульта на защите и управлять показом будет другой человек.
        \item<1-> Оценка трудоемкости случайного поиска экстремума со
        случайной начальной точкой.
        \item<2-> Аналитическое решение вспомогательной оптимизационной
        задачи.
        \item<3-> Моделирование равномерного распределения в $d$-мерном
        шаре.
    \end{itemize}
\end{frame}

\begin{frame}
    \frametitle{Поиск: основные определения и обозначения}

    \begin{itemize}
    %\alert - это выделение в стиле выбранной "темы" (т.е. оформления) слайдов
        \item<1-> \alert{Пространство оптимизации\,:} тор
        $\mathbb{I}^d=(0,1]^d$ с равномерной метрикой $\rho_d$.
        \item<2-> \alert{Целевая функция}
        $f:\mathbb{I}^d\mapsto\mathbb{R}$
        ограничена, измерима и
        \begin{enumerate}
            \item принимает максимальное значение в единственной точке
            $x_0$,
            \item непрерывна в точке $x_0$,
            \item неравенство $\sup\{f(x):x\in S_r^c(x_0)\}<f(x_0)$ верно
            для любого $r>0$.
        \end{enumerate}
        \item<3-> \alert{Случайный поиск\,:}
        \begin{center} {Алгоритм}
        \end{center}
        \begin{description}
            \item[Шаг~1.] $\xi_0 \gets x$; $i \gets 1$.
            \item[Шаг~2.] $\eta \gets P(\xi_{i-1},\cdot\,)$.
            \item[Шаг~3.] Если $f(\eta)\geq f(\xi_{i-1}),$ то $\xi_{i}
            \gets \eta$, иначе $\xi_{i} \gets \xi_{i-1}$.
            \item[Шаг~4.] Если $i < n$, то $\big(i \leftarrow i + 1$ и
            перейти к шагу 2$\big)$, иначе --- STOP.
        \end{description}
        \item<4-> Вероятность $P(\mathbf{x},\,\cdot\,)$ обладает
        \alert{симметричной} плотностью
        \begin{gather*}
            p(\mathbf{x},\mathbf{y})=g(\rho_d(\mathbf{x},\mathbf{y})),
        \end{gather*}
        где $g$ монотонно убывает.
    \end{itemize}
\end{frame}

\begin{frame}
    \frametitle{Поиск: цель и характеристики случайного поиска}
    {\alert{Цель поиска\,:}} --- попадание в множество
    \begin{gather*}
        M_\varepsilon=\{x\in S_\varepsilon(x_0):f(x)>f(y)\text{ для
        }y\in S_\varepsilon^c(x_0)\},
    \end{gather*}
    где $S_\varepsilon(x_0)$ --- шар радиуса $\varepsilon$ с центром в
    $x_0$.

    Характеристика качества функции --- коэффициент асимметрии:
    \begin{gather*}
        F^f(r)=mes_d(M_r)/mes_d(S_r(x_0)).
    \end{gather*}

    {\alert{Невырожденная}} функция: $F^f(r)\geq \theta>0$.

    {\alert{Трудоемкость}} поиска: $\mathsf{E}_x\tau_\varepsilon$,
    где $\tau_\varepsilon=\min\{i\geq0:\xi_i\in M_\varepsilon\}$.  $x$
    --- начальная точка поиска.

    {\alert{Общая проблема\,:}} оценить/уменьшить трудоемкость
    поиска.
\end{frame}

\begin{frame}
    \frametitle{Поиск: известные результаты}
    \alert{Результаты А.С. Тихомирова.}
    \begin{itemize}
        \item Имеют место неравенства
        \begin{gather}
            \label{eq:Tih1}
            C|\ln\varepsilon|\leq \mathsf{E}_x\tau_\varepsilon\leq
            I(\delta(x),\varepsilon;f,g).
        \end{gather}
        \item Существует $g_{opt}$, доставляющая минимум правой части
        (\ref{eq:Tih1}).
        \item Для невырожденных целевых функций существуют такие $g$,
        что
        \begin{gather*}
            \mathsf{E}_x\tau_\varepsilon\leq C(f,d)\ln^2(\varepsilon).
        \end{gather*}
        \item Если $F^f\equiv\theta$, то $g_{opt}$ и
        $I(\delta(x),\varepsilon;f,g_{opt})$ находятся явно, причем
        $g_{opt}$ не зависит от $\theta$.
    \end{itemize}

    \alert{Проблемы\,:}
    \begin{itemize}
        \item $g_{opt}$ зависит от $x$, то есть от взаимного
        расположения начальной точки поиска и точки экстремума. А оно
        на практике неизвестно.
        \item Результаты, относящиеся к классам функций вида
        $\{f:\,F^f(r)\geq \theta>0\}$.
    \end{itemize}
\end{frame}

\begin{frame}
    \frametitle{Поиск: случайная начальная точка. Результаты}
    \alert{Основные результаты}.

    Пусть начальная точка поиска $\xi$ равномерно распределена в
    $\mathbb{I}^d$. Тогда
    \begin{itemize}
        \item $ \mathsf{E}\tau_\varepsilon\leq J(\varepsilon;f,g), $
        где $J(\varepsilon;f,g)$ явно представлено в интегральной
        форме.
        \item Если $F^f(r)\geq \theta>0$, то при $\varepsilon<0.25$
        \begin{equation}
            \label{eq:J_id}
            J(\varepsilon;f,g)\leq J_\theta(\varepsilon;g)=
            \frac{1}{\phi}\left(
                d\int_{2\varepsilon}^{0.5}\frac{1}{z^{d+1}g(z)}\,\mathrm{d}z+
                \frac{2^d-\phi}{g(0.5)}\right).
        \end{equation}
        \item Правая часть неравенства (\ref{eq:J_id}) достигает
        своего минимума при
        \begin{gather}
            g(r)= g_{\rm opt}(r)= \left(d\ln
                (\beta/\alpha\varepsilon)-(2\beta)^{-d}\right)^{-1}
            \begin{cases}
                (\alpha\varepsilon)^{-d}, &
                \text{при $0< r\leqslant \alpha\varepsilon$,} \\
                r^{-d}, &
                \text{при $ a\varepsilon<r\leqslant\beta$,} \\
                \beta^{-d}, &
                \text{при $\beta<r\leqslant0.5$,} \\
            \end{cases}
            \label{eq:g:optimal:1}
        \end{gather}
        если $\varepsilon<a(d,\theta)$. Иначе $g_{opt}\equiv 1$.
    \end{itemize}
    \alert{Замечание\,.} Величина $J_\theta(\varepsilon;g_{opt})$ и
    постоянные $a$, $\alpha$ и $\beta$ выписываются явно.
\end{frame}

\begin{frame}
    \frametitle{Оптимизационная задача: постановка}
    \alert{Постановка задачи\,:}\, ставится задача минимизации
    функционала
    $$
    \mathcal{J}_{u,v}(w)=\int_u^v\frac{h^2(r)}{w(r)}+\frac{c}{w(1)},
    $$
    (где $0<u<v\leq1$, $c>0$ и $h\in\mathbb{L}^2(u,v)$ ---
    неотрицательная функция) в классе невозрастающих строго
    положительных непрерывных слева плотностей $w$.

    \alert{Мотивация\,:} см. формулу (\ref{eq:J_id}).

    \alert{База\,:} А.С. Тихомиров для случая $v=1$.

    \alert{Результаты А.С. Тихомирова} (краткая сводка):
    \begin{itemize}
        \item Теорема существования $w_{opt}$.
        \item Анализ структуры $w_{opt}$.
        \item Явный вид $w_{opt}$ в случае $v=1$, когда функция $h$
        гладкая и строго убывает (2 параметра).
    \end{itemize}

\end{frame}

\begin{frame}
    \frametitle{Оптимизационная задача: результаты}
    \alert{Полученные результаты} для случая произвольной $h$ и
    $v\leq 1$:

    \begin{itemize}
        \item Теорема существования $w_{opt}$.
        \item Анализ структуры $w_{opt}$.
        \item Вид $w_{opt}$ в случае, когда функция $h$ является
        непрерывной и строго убывает (3 параметра).
    \end{itemize}

    \alert{Общий вид \,:}

    \begin{equation*}
        % \label{eq:w^1}
        w_{\rm opt}(r)=w_{b,d,\theta}(r)=
        \frac{1}{\lambda_{b,d,\theta}}
        \begin{cases}
            h(b),   & \text{при $r\in(0,b]$,} \\
            h(r),   & \text{при $r\in(b,d]$,} \\
            h(d),   & \text{при $r\in(d,v]$,} \\
            \theta, & \text{при $r\in(v,1]$} \\
        \end{cases}
    \end{equation*}
    с $u<b\leq d\leq v$, $\theta\leq h(d)$.

    \vspace{1mm} \alert{Техника:\,} А.С.Тихомиров.
\end{frame}

\begin{frame}
    \frametitle{Моделирование: алгоритм И.В.Романовского}
    \alert{Задача\,:}\, моделирование
    р. р. в единичном $d$-мерном шаре\\

    \alert{База\,:} алгоритм (и реализация) Л.А.Евдокимова и И.В.Романовского.\\
    \alert{Идея\,:}
    \begin{itemize}
        \item Шар {\bluetext{большого}} радиуса $R$ ($R^2$ --- целое).
        \item Покрытие шара \bluetext{единичными} кубами
        (целочисленные вершины)
        \begin{gather*}
            c_d(t)=\{x|t_j\leq x_j<t_j+1,, j=1,\ldots,d\}.
        \end{gather*}
        \item \bluetext{Параметризация} кубов с помощью
        векторов $t=(t_1,\ldots,t_d)$.
        \item Моделирование номера $i$ куба.
        \item \bluetext{Сопоставление} номера $i$ кубу $t^{(i)}$
        (метод Уолкера --- $d$ раз).
        \item Моделирование р. р. в кубе $t^{(i)}$ и проверка принадлежности  шару.\\
        Если ``\alert{да}'', то деление полученного вектора на $R$.
    \end{itemize}

    \alert{Параметры} алгоритма: $d$, $R^2$.
\end{frame}

\begin{frame}
    \frametitle{Моделирование: ограничения и затраты}
    \alert{Ограничения\,:}\, число кубов $\leq 2^{31}-1$ (тип long).\\
    \alert{Объем памяти\,:}\, необходимый объем памяти $\sim 8dR^3$
    байт.

    \alert{Результат:} трудоемкость отбора при разных ограничениях
    на память.

    \begin{center}
        \begin{tabular}{|c|c|c|c|c|}
            \hline
            $d$& $C_{min}$ & 10 Mb & 5 Mb &1 Mb\\ \hline
            2& $\approx$1& 1.01 & 1.01 & 1.03\\ \hline
            3& $\approx$1&1.03&1.03&1.06\\ \hline
            % 4&1.01&1.11&1.06&1.05\\ \hline
            5& 1.05 & 1.07 & 1.09& 1.17 \\ \hline
            % 6& 1.12 & 1.24&1.12&1.12\\\hline
            7&1.25&1.25&1.25&1.33\\\hline
            10& 2.31 & 2.31&2.31 &2.31\\ \hline
            20& 229&229&229&229\\\hline
        \end{tabular}
    \end{center}

    \alert{Проблема\,:}\, При ограничении на трудоемкость отбора
    $<2$ получаем ограничение $<10$ на
    размерность $d$.\\
\end{frame}

\begin{frame}
    \frametitle{Моделирование: результаты}
    \alert{Модификации\,:}\,
    \begin{itemize}
        \item Хранение целых в виде частного и остатка при делении на
        $2^{32}-1$ (структура superlong).\\ Ограничение: число кубов
        $<2^{32}(2^{32}-1)-1$.
        \item Оптимизация хранения массивов (выигрыш $\sim 13R^3$
        байт).
        \item Частичное использование типа long (выигрыш $\asymp dR^3$
        байт).
    \end{itemize}

    \alert{Результат:} трудоемкость отбора резко падает при больших
    $d$.

    \begin{center}
        \begin{tabular}{|c||c|c||c|c||c|c|}
            \hline
            &  \multicolumn{2}{c||}{10 Mb}&  \multicolumn{2}{c||}{5 Mb}&  \multicolumn{2}{c|}{1 Mb}  \\\hline
            d&$C_{long}$&$C_{slong}$&$C_{long}$&$C_{slong}$&$C_{long}$&$C_{slong}$\\ \hline
            2& 1.01& $\approx1$ &1.01& $\approx1$&1.03 & $\approx1$  \\ \hline
            3& 1.03& 1.02&1.03&1.02&1.06&1.04\\ \hline
            5& 1.07&1.06&1.09&1.08&1.17&1.14\\\hline
            7&1.25&1.13&1.25& 1.17&1.25&1.29\\\hline
            10&2.31&1.29  &2.31&1.37&2.31&1.69\\\hline
            % 15&115.2&2.19&2.11&2.01\\\hline
            20&229&5.27  &229&5.27&229&7.34\\\hline
        \end{tabular}
    \end{center}
\end{frame}

\begin{frame}
    \frametitle{Моделирование: тестирование и программа}
    \alert{Тестирование.}\\
    \bluetext{Статистика критерия:}\\
    Пусть $(\xi_1,\ldots,\xi_d)$ р.р. в единичном $d$-мерном шаре
    $B_d(1)$.  Тогда \begin{gather*}
        \left(\frac{\xi^2_{i+1}+\ldots+\xi^2_d}{1-\xi^2_1-\ldots-\xi_i^2}\right)^{d-i}
    \end{gather*}
    р.р. на $(0,1)$ для любого $0\leq i<d$.\\
    \bluetext{Критерии:} Колмогорова и $\chi^2$ с 20
    интервалами. $N=1000$.

    \alert{Программа\,:}\,
    \begin{itemize}
        \item Реализация алгоритма с модификациями для размерности
        $d\leq20$.
        \item Выбор параметров алгоритма согласно заданному
        ограничению по памяти
        \item Тестирование сгенерированной выборки
    \end{itemize}
\end{frame}

\end{document}
