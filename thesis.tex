\documentclass{matmex-diploma-custom} 


\newcommand{\Expect}{\mathbb E}
\newcommand{\PRob}{\mathbb P}
\newcommand{\leqs}{\leqslant}
\newcommand{\geqs}{\geqslant}
\newcommand{\eps}{\varepsilon}

\newtheorem{theorem}{Теорема}
\newtheorem{note}{Замечание}
\newtheorem{cons}{Следствие}
\newtheorem{define}{Определение}
\newtheorem{lemma}{Лемма}

\newtheoremstyle{named}{}{}{\itshape}{}{\bfseries}{.}{.5em}{#1 #3}
\theoremstyle{named}
\newtheorem*{namedtheorem}{Теорема}
\newtheorem*{namedpropo}{}


\emergencystretch=10pt % для обработки плохих строк
\clubpenalty=10000 % избежание одиночных слов в начале или конце страницы
\widowpenalty=10000 % для того же
\binoppenalty=10000 % неразрывность формул
\relpenalty=10000 % и это туда же


\begin{document}

\filltitle{ru}{
    chair              = {Кафедра теории вероятностей и математической статистики},
    title              = {Рост двудольных графов с реберным приоритетом},
    type               = {diploma},
    position           = {студента},
    group              = 512,
    author             = {Ерохин Станислав Евгеньевич},
    supervisorPosition = {кандидат физ.-мат. наук, доцент},
    supervisor         = {Якубович Ю.\,В.},
    reviewerPosition   = {кандидат физ.-мат. наук, доцент},
    reviewer           = {Валландер С.\,С.},
    chairHeadPosition  = {доктор физ.-мат. наук, профессор},
    chairHead          = {Никитин Я.\,Ю.},
}
\filltitle{en}{
    chair              = {Chair of Probability Theory and Mathematical Statistics},
    title              = {Bipartite graph growth with edge priority},
    author             = {Stanislav Erokhin},
    supervisorPosition = {Associate Professor},
    supervisor         = {Yuri Yakubovich},
    reviewerPosition   = {Associate Professor},
    reviewer           = {Sergej Vallander},
    chairHeadPosition  = {Professor},
    chairHead          = {Yakov Nikitin},
}
\maketitle
	
%%%%%%------------------Оглавление	
\renewcommand{\contentsname}{Оглавление}
\tableofcontents
\newpage

%%%%%%------------------Введение
\section*{Введение}
В 2001 году был создан BitTorrent --- сетевой протокол для кооперативного обмена файлами через Интернет.
Протокол быстро набрал популярность среди пользователей и сейчас более 30\% трафика приходится именно на этот протокол.
Опишем схему его действия.

Пусть есть несколько пользователей, которые хотят получить файл. Дальше будем называть их клиентами.
Изначально, этот файл есть только у одного клиента. 
Для удобства, файл разбивается на фрагменты, которых обычно несколько тысяч.
В начале клиент, у которого есть все фрагменты, раздаёт другим клиентам разные фрагменты этого файла.
Затем клиенты начинают обмениваться этими фрагментами по некоторым правилам, 
которые описаны в спецификации протокола \cite{spec}.
Есть несколько модификаций этого протокола, часть из которых рассмотрена в \cite{cool}.
В основе всех этих модификаций лежат некоторые случайные модели выбора клиента, 
с которым будет производиться обмен данными и очередного фрагмента для скачивания. 
Все эти модели достаточно сложны, поэтому остаются плохо изученными с теоритической точки зрения.

Чтобы хоть как-то изучить эти сложные модели, вводят ряд упрощений.
Например, в статье \cite{cool}, рассматривается система, состоящая только из клиентов, 
тем самым теряется информация, касающаяся фрагментов.
В реальной жизни может случиться так, что какой-то фрагмент есть только у нескольких клиентов, которые, 
по каким-то причинам, пропали, тем самым файл целиком стал недоступен.
По-этому, информация о фрагментах тоже важна. 
В основном же изучение моделей происходит путём моделирования \cite{unavalib}.


В этой работе мы упростим модель следующем образом: мы рассмотрим двудольный граф $G$, 
одна из долей будет составлена из клиентов, у которых в начале нет файла, а вторая -- из фрагментов. 
Ребро будет проводиться в том случае, когда клиент будет получать соответствующий фрагмент.
Также, мы будем считать, что время у нас дискретно, т.е. рассматривать только те моменты, когда будет появляться очередное ребро.
Соответственно граф на $m$-ном шагу мы обозначим за $G_m$.
Для изучения 'редкости' фрагментов, мы зададим специальную функцию $Q(G)$, которая будет выдавать минимальную степень фрагмента.
Мы будем исследовать её поведение.

Среди исследований на схожую тематику, нет работ посвящённых изучению таких растущих графов.
Однако, стоит отметить работы, посвящённые изучению моделей цитирования \cite{price}, \cite{newman}. 
В них рассматриваются растущие ориентрированные графы, в которых вершинам соответствуют статьи, а ребрам --- ссылки между ними. 
Публикация новой статьи соответствует добавлению в граф новой вершины.
Из новой вершины проводятся рёбра, в какие-то уже существующие, при этом между существующими вершинами новых ребер не появляется.
В нашем же случае, количество вершин не изменяется, а изменяются только ребра между уже существующими вершинами.

Основной нашей задачей будет нахождение такого момента $m$, что к этому моменту гарантированно каждый фрагмент 
будет скачан заданной долей клиентов $q$, где $q \in (0, 1)$. 
Для разных моделей роста графа необходимые моменты $m$ будут различны, 
поэтому нас будут интересовать не только оценки на $m$ в разных моделях, но и то, какая из моделей оптимальнее. 
(То есть в какой из моделей необходимый момент $m$ меньше.) 
Выбрать оптимальную из всех возможных моделей не представляется возможным, 
поэтому мы выбрали три наиболее естественных, на наш взгляд, и исследовали их. 
Для каждой из рассмотренных моделей мы получили верхние и нижние оценки на число шагов с заданными вероятностями ошибок. 
В результате мы выяснили, какая из рассмотренных моделей лучше остальных.

Стоит отметить, что нас будет интересовать не предельное поведение момента $m$, а точные оценки при заданном количестве клиентов
и фрагментов. Также мы будем предполагать, что фрагментов хотя бы $100$, а клиентов хотя бы $10$.

%%%%%%------------------Основные определения
\section*{Основные определения}
Итак, мы будем иметь дело только с двудольными графами, доли которых мы условно назовём клиенты и фрагменты.
Ребро между некоторым клиентом и фрагментом будет означать, что данный клиент скачал данный фрагмент.
Обозначим количество клиентов за $n$, а количество фрагментов за $N$. 
В рамках данной работы, будем предполагать, что $N \geqs 100$, a $n > 10$.
Заметим, что максимальное возможное количество рёбер в таком графе $nN$.

Пусть дан произвольный двудольный граф $G$. Определим для него несколько функций:
\begin{align*}
&E(G) \text{ --- количество рёбер}\\
&Q(G) \text{ --- минимальная степень фрагмента}
\end{align*}
Нас будет интересовать, как ведёт себя функция $Q(G)$ при различных моделях роста графа $G$.

\begin{define}
Некоторое свойство графа будем называть монотонным, если оно сохраняется при добавлении рёбер.
\end{define}

\begin{note} \label{note1}
Пусть $k$ -- натуральное число. Тогда свойство $Q(G) > k$ является монотонным свойством.
\end{note}

Также сформулируем некоторые известные факты, которыми будем пользоваться.

\newpage

\begin{namedtheorem}[Чернова] (Для бернуллиевских случайных величин) \label{Cher} 

Пусть у нас есть $X_1, X_2, X_3, \dots X_n$ -- 
независимые бернуллиевские случайные величины(не обязательно одинаково распределенные). Пусть $X = \sum\limits_{i = 1}^n X_i$ и
$\mu = \Expect X$, также пусть $\delta > 0$. Тогда выполняются следующие неравенства:

\begin{align}\label{ch_1}
\PRob[X < (1-\delta)\mu] &< \exp\left(- \frac{\mu \delta^2}{2} \right)
\\\label{ch_2}
\PRob[X > (1+\delta)\mu] &< \exp\left(- \frac{\mu \delta^2}{4} \right) \text{, при } \delta < 2e - 1
\end{align}
\end{namedtheorem}
Доказательство этой теоремы можно найти в статье \cite{chernov}.

\begin{namedtheorem}[Чернова] (Для пуассоновской случайной величины)

Пусть $X$ -- пуассоновская случайная величина с показателем $\lambda$, тогда верны следующие неравенства:
\begin{align} \label{gr_1}
\PRob(X \geqs x) \leqs \frac{e^{-\lambda}(e\lambda)^x}{x^x}, \text{ для всех } x > \lambda 
\\
\label{gr_2}
\PRob(X \leqs x) \leqs \frac{e^{-\lambda}(e\lambda)^x}{x^x}, \text{ для всех } x < \lambda 
\end{align}
\end{namedtheorem}
Доказательство этого утверждения можно найти в книге \cite{chernov_gr}.

\medskip

Также напомним, что такое $I_x(a,b)$ -- регуляризованная неполная бета-функция. 
Мы её будем использовать для натуральных $a$, $b$, и $x \in [0,1]$.
Определена она следующим образом:
\begin{equation} \label{i_x} \begin{aligned}
I_x(a,b) = \frac{B_x(a,b)}{B_1(a,b)}, \text{ где } 
\\
B_x(a,b) = \int_0^x t^{a-1} (1-t)^{b-1} dt.
\end{aligned}\end{equation}
Для этой фунции верно, что $I_x(a,b) = 1 - I_{1-x}(b, a)$.


%%%%--------------------------------модель 1
\newpage
\section*{Модель 1. Равномерный выбор рёбер.}
В этой модели, граф будет расти следующим образом: на каждом шагу мы равномерно выбираем произвольное, ещё не проведённое ребро 
и проводим его. Обозначим за $G_m$ граф, который получается на $m$-том шагу. Ясно, что $E(G_m) = m$.
Выберем некоторое число $q \in (0, 1)$ и зададимся вопросом, насколько велико должно быть $m$, чтобы выполнялось: $Q(G_m) > qn$.
К сожалению, исследовать граф $G_m$ напрямую не очень удобно, поэтому мы создадим некоторый граф, который будет похож на граф $G_m$.

Создадим граф $H_p$, где $p \in [0, 1]$, который будет построен по следующему правилу: 
разыгрываем $nN$ бернуллиевских случайных величин $I_{1,1}, I_{1,2}, \dots I_{n,N}$ с показателем $p$ и проводим ребро от $k$-го клиента к $K$ фрагменту, 
если $I_{k, K} = 1$.
Заметим, что случайный граф $G_m$ устроен следующим образом: все графы, в которых ровно $m$ рёбер, выпадают с одинаковой вероятностью.
А если взять граф $H_p$ при условии того, что $E(H_p) = m$, то тогда он устроен так же, как и $G_m$.
Это позволит нам доказать следующую лемму:

%лемма 1
\begin{lemma} \label{l1}
Пусть $A$ - множество всех графов, который обладают некоторым монотонным свойством, а $\delta \in (0,1)$.
Тогда, если $\PRob( H_p \in A) \geqs 1 - \eps$, то
\begin{equation} \label{l1_1}
\PRob(G_{\lceil pnN(1+\delta) \rceil} \in A) > 1 - \frac{\eps}{1 - \exp\left(-\frac{\delta^2}{4}pnN\right)},
\end{equation}
a если $\PRob( H_p \in A) \leqs \eps$, то
\begin{equation}\label{l1_2}
\PRob(G_{\lfloor pnN(1-\delta) \rfloor} \in A) < \frac{\eps}{1 - \exp\left(-\frac{\delta^2}{2}pnN\right)}.
\end{equation}
\end{lemma}

\begin{proof}
Будем доказывать неравенство (\ref{l1_1}). Заметим, что 
\begin{equation}
E(H_p) = \sum_{k=1}^n \sum_{K=1}^N I_{k, K},
\end{equation}
где $I_{k,K}$ -- бернуллиевские случайные величины с показателем $p$.
Применим к $E(H_p)$ теорему Чернова, где $\mu = \Expect\big(E(H_p)\big) = pnN$, а $\delta$ взята из условия леммы.
Используя неравенство (\ref{ch_2}), получим следующее:
\begin{equation}\label{l1_3}
\PRob\Big( E(H_p) \leqs pnN(1+\delta) \Big) > 1 - \exp\left(-\frac{\delta^2}{4}pnN\right).
\end{equation}
Заметим, что $m = \lceil pnN(1+\delta) \rceil \geqs pnN(1+\delta)$, а значит
\begin{equation}\label{l1_4}
\PRob\Big( E(H_p) \leqs m \Big) \geqs \PRob\Big( E(H_p) \leqs pnN(1+\delta) \Big).
\end{equation}
Положим $B = \PRob\Big( H_p \in A~\big|~E(H_p) \leqs m \Big)$. Заметим, что
\begin{equation}\label{l1_big}
\begin{aligned}
B = \frac{
		\PRob\Big( H_p \in A~\bigcap~E(H_p) \leqs m \Big)
	}{
		\PRob\big(E(H_p) \leqs m \big)
	} 
&\geqs 
	\frac{
		\PRob\big(E(H_p) \leqs m \big) 
			-
		\PRob\big(H_p \not\in A \big)
	}{
		\PRob\big(E(H_p) \leqs m\big)
	}
=
\\
=
	1 
		-
	\frac{
		\PRob\big(H_p \not\in A \big)
	}{
		\PRob\big(E(H_p) \leqs m\big)
	}
&>
	1
		-
	\frac{\eps}{
		1 - \exp\left(-\frac{\delta^2}{4}pnN\right)
	},
\end{aligned}\end{equation}
где в последнем переходе мы воспользовались неравенствами (\ref{l1_3}), (\ref{l1_4}) 
и неравенством $\PRob\big(H_p \not\in A \big) < \eps$, которое очевидно из условия леммы.
Теперь докажем, что $\PRob\big(G_{\lceil pnN(1+\delta) \rceil} \in A\big) \geqs B$.

Рассмотрим граф $H_p$ при условии того, что $E(H_p) \leqs m$ и обозначим его $H_p^m$. 
Заметим, что этот граф удовлетворяет свойству $A$ с вероятностью $B$, а $E(H_p^m) \leqs m$.
Теперь мы несколько раз добавим ребер в этот граф, пока в нем не станет ровно $m$ ребер. 
При добавлении рёбер мы будем выбирать их равномерно среди ещё не проведенных.
\\
Заметим, что мы получим случайный граф, в котором ровно $m$ ребер. Несложно понять, 
что этот граф по распределению совпадает с $G_m$, т.к. мы нигде не различали рёбра между собой, а значит они перестановочны.
При этом мы знаем, что граф $H_p^m$, с которого мы начинали, обладал свойством $A$ с вероятностью $B$, 
а значит $\PRob(G_m \in A) \geqs B$. Применяя неравенство (\ref{l1_big}) получаем (\ref{l1_1}).

Теперь докажем (\ref{l1_2}). Пусть $l = \lfloor pnN(1-\delta) \rfloor$. Аналогично неравенствам (\ref{l1_3}) и (\ref{l1_4}) получаем
\begin{equation} \label{l1_5}
\PRob\big( E(H_p) \geqs l \big) > 1 - \exp\left(-\frac{\delta^2}{2}pnN\right).
\end{equation}
После этого получаем аналог неравенства (\ref{l1_big}):
\begin{equation}\label{l1_big2}
\begin{aligned}
	\PRob\Big( H_p \not\in A~\big|~E(H_p) \geqs l \Big)
=
	\frac{
		\PRob\Big( H_p \not\in A~\bigcap~E(H_p) \geqs l \Big)
	}{
		\PRob\big(E(H_p) \geqs l \big)
	} 
\geqs
\\
\geqs
	\frac{
		\PRob\big(E(H_p) \geqs l \big) 
			-
		\PRob\big(H_p \in A \big)
	}{
		\PRob\big(E(H_p) \geqs l\big)
	}
>
	1
		-
	\frac{\eps}{
		1 - \exp\left(-\frac{\delta^2}{2}pnN\right)
	},
\end{aligned}\end{equation}
После этого остаётся доказать, что $\PRob(G_l \not\in A) \geqs \PRob\Big( H_p \not\in A~\big|~ E(H_p) \geqs l \Big)$.
Это делается аналогично предыдущему случаю, только на сей раз мы не добавляем рёбра, а равномерно выкидываем их.
\end{proof}

\bigskip

Только что доказанная лемма позволяет нам оценивать $\PRob\big(Q(G_m) \geqs qn\big)$ через $\PRob\big(Q(H_p) \geqs q n\big)$.
Ясно, что рассматривать $q \geqs p$ не очень интересно, так как ребер в графе~$H_p$ в среднем $p nN$, 
а значит, самое лучшее, на что мы можем рассчитывать, что каждый фрагмент скачан $p  n$ клиентами.

Итак, $q < p$. Рассмотрим вероятность того, что первый фрагмент скачали хотя бы~$qn$ клиентов, 
т.е. $\PRob\left(\sum\limits_{i=1}^n I_{i,1} \geqs qn\right)$. Пусть $\delta = 1 - \frac{q}{p}$, тогда:
\begin{align*}
\PRob\left(\sum_{i=1}^n I_{i,1} \geqs qn\right) = \PRob\left(\sum_{i=1}^n I_{i,1} \geqs (1 - \delta) pn\right) 
= 1 - \PRob\left(\sum_{i=1}^n I_{i,1} < (1 - \delta) pn\right).
\end{align*}
Применяя теорему Чернова (\ref{ch_1}), получаем:
\begin{align}
&\PRob\left(\sum_{i=1}^n I_{i,1} < (1 - \delta) pn\right) < \exp\left(- \frac{pn \delta^2}{2} \right),
\\
&\PRob\left(\sum_{i=1}^n I_{i,1} \geqs qn\right) > 1 - \exp\left(- \frac{pn \delta^2}{2} \right).
\end{align}

Теперь посмотрим, что мы знаем, про вероятность того, что все фрагменты скачены хотя бы $qn$ клиентами.
Заметим, что $I_{i,j}$ независимы, а значит и то, что происходит с первым фрагментом не зависит от того, 
что происходит с остальными фрагментами. Пользуясь этим получим:
\begin{equation}\label{mod1_pr}\begin{aligned}
&\PRob\big(Q(H_p) \geqs q n\big) = \left( \PRob\left(\sum_{i=1}^n I_{i,1} \leqs qn\right)  \right)^N >
\\
&\left( 1 - \exp\left(- \frac{pn \delta^2}{2} \right) \right)^N > 1 - N \cdot \exp\left(- \frac{pn \delta^2}{2} \right) 
\end{aligned}\end{equation}
где $\delta = 1 - \frac{q}{p}$. Всё вышесказанное позволит нам доказать следующую теорему:

%теорема 1
\begin{theorem}\label{t1}
Пусть $q, \sigma \in (0, 1)$. Возьмем 
\begin{align}
& m = \min(nN, \lceil pnN + 2\sqrt{pnN} \rceil), \quad \text{ где} \\
& p \geqs q + c + \sqrt{c^2+2qc}, \quad c = \frac{\ln(2N) - \ln(\sigma)}{n}.
\end{align}
Тогда в графе $G_m$ с вероятностью, большей, чем $1 - \sigma$, все фрагменты будут скачены хотя бы $qn$ клиентами:
\begin{equation}
\PRob\big(Q(G_m) \geqs qn\big) > 1 - \sigma.
\end{equation}
\end{theorem}

\begin{proof}
Для начала заметим, что если получится, что $pnN + 2\sqrt{pnN} \geqs nN$ то утверждение очевидно, 
поэтому далее мы будем считать, что $pnN + 2\sqrt{pnN} < nN$.

Сначала мы докажем, что 
\begin{equation}\label{t1_1}
\PRob\big(Q(H_p) \geqs qn \big) > 1 - \frac\sigma{2}.
\end{equation}
Из неравенства (\ref{mod1_pr}) мы знаем, что 
\begin{equation} \label{t1_t3_1}
\PRob\big(Q(H_p) \geqs qn \big) > 1 - N \cdot \exp\left(- \frac{n (p-q)^2}{2p} \right).
\end{equation}
Заметим, что $(p-q)^2 - 2pc = p^2 -2(q + c)p + q^2 \geqs 0$, а значит:
\begin{equation}
1 - N \cdot \exp\left(- \frac{n (p-q)^2}{2p} \right) = 1 - N \cdot \exp(-nc) = 1 - \frac\sigma{2}.
\end{equation}
Тем самым мы доказали неравенство (\ref{t1_1}).

Осталось воспользоваться леммой \ref{l1}, а если точнее, то неравенством (\ref{l1_1}). 
В качестве $\delta$ и $\eps$ возьмём $\frac{2}{\sqrt{pnN}}$ и $\frac\sigma{2}$ соответственно, 
а в качестве множества $A$ -- все графы, для которых $Q(G) \geqs qn$.
По замечанию \ref{note1}, такое свойство является монотонным.
Также вспомним, что мы предположили, что $N \geqs 100$, а значит $pnN > 2cnN> 2N\ln(2N) > 4$, 
следовательно, $\delta \in (0,1)$. 
Осталось заметить, что 
\begin{equation}
1 - \exp\left(-\frac{\delta^2}{4}pnN\right) = 1 - \frac{1}{e} > \frac{1}{2},
\end{equation}
что и завершает доказательство.
\end{proof}

\bigskip

Только что доказанная теорема позволяет находить необходимое количество проведённых ребер, чтобы выполнялось $Q(G_m) > qn$.
Попробуем привести оценку в другую сторону -- т.е. понять, какого количества точно не хватит.

Вернёмся к графу $H_p$, где $p \in [0,1]$ и вспомним, что ребро от $k$-го клиента к $K$-му фрагменту проводится тогда и только тогда
когда $I_{k,K} = 1$, где $I_{k,K}$ -- независимые бернуллиевские случайные величины с показателем $p$. 
Посчитаем вероятность того, что первый фрагмент скачали как минимум $qn$ клиентов:
\begin{equation}
\PRob\left(\sum_{i=1}^n I_{i,1} > qn\right) 
	= 
1 - \PRob\left(\sum_{i=1}^n I_{i,1} \leqs \lfloor qn \rfloor\right) 
	= 
1 - \PRob\big(Bin(n, p) \leqs l\big),
\end{equation}
где $l = \lfloor qn \rfloor$. Заметим, что для биномиального распределения мы знаем функцию распределения, 
а именно $\PRob\big(Bin(n, p) \leqs l\big) = I_{1-p}(n-l, l+1)$, где $I_x(a,b)$ -- регуляризованная неполная бета-функция (\ref{i_x}).

Вспомнив, что $I_{k,K}$ -- независимые, посчитаем
\begin{equation}\label{n21}
\PRob(Q(H_p) > qn) = \big(1 - I_{1-p}(n-l, l+1)\big)^N = \big(I_{p}(l+1, n-l)\big)^n.
\end{equation}
Теперь всё готово, для того, чтобы доказать аналог теоремы \ref{t1}.

% теорема 2
\begin{theorem}\label{t2}
Пусть $q, \sigma \in (0, 1)$, $l = \lfloor qn \rfloor$. Возьмем 
\begin{align}
& m = \max(0, \lfloor pnN - \sqrt{2pnN} \rfloor), \quad \text{ где} \\
& p \leqs \max\left(p :  I_p(l+1, n-l) \leqs \sqrt[n]{\frac\sigma{2}} \right).
\end{align}
Тогда в графе $G_m$ с вероятностью, меньшей, чем $\sigma$, все фрагменты будут скачены хотя бы $qn$ клиентами:
\begin{equation}
\PRob\big(Q(G_m) \geqs qn\big) < \sigma.
\end{equation}
\end{theorem}


\begin{proof}
Как и в теореме \ref{t1}, предполагаем, что $pnN - \sqrt{2pnN} > 0$.
Заметим, что $I_x(a,b)$ -- монотонная по $x$ функция, а значит $I_p(l+1, n-l)^n \leqs \frac\sigma{2}$. 
Из неравенства (\ref{n21}) получаем, что
\begin{equation}
\PRob(Q(H_p) > qn) \leqs \frac\sigma{2}.
\end{equation}

Теперь воспользуемся леммой \ref{l1}, на сей раз неравенством (\ref{l1_2}).
Возьмём в качестве $A$ множество всех графов, для которых $Q(G) > qn$.
Также возьмём $\eps = \frac\sigma{2}$, $\delta = \sqrt{\frac{2}{pnN}}$.
Предположение $pnN - \sqrt{2pnN} > 0$ означает, что $\delta < 1$.
Заметим, что $1 - \exp\left(-\frac{\delta^2}{2}pnN\right) = 1 - \frac{1}{e} > \frac{1}{2}$, откуда получим, что
\begin{equation}
\PRob\big(G_{\lfloor pnN - \sqrt{2pnN} \rfloor} \in A\big) 
	< 
\frac{\frac\sigma{2}}{1 - \exp\left(-\frac{\delta^2}{2}pnN\right)}
	<
\sigma.
\end{equation}
\end{proof}

Как видно выше, оценка того, сколько ребер не хватит, для того, 
чтобы каждый фрагмент был скачан хотя бы $qn$ клиентами, получилась довольно сложной.
Тем не менее, поскольку $I_x(a, b)$ выражается в виде частного интегралов, 
с помощью этой теоремы мы можем делать численные оценки для реальных данных.

\newpage
% модель 2
\section*{Модель 2. Выбор степеней клиентов.}
В этой моделе $G_m$ будет устроен немного другим образом. На сей раз, за каждый шаг, мы будем равномерно выбирать клиента, 
у которого степень ещё не максимальная, а затем проводить ещё не проведённое ребро от выбранного клиента равномерным образом.
Заметим, что $E(G_m) = m$. Как и в прошлой моделе $G_m$ не очень удобен для исследования, поэтому мы построим граф $H_p$, где $p \in [0,1]$.

Граф $H_p$ строится так: выбираем $J_1, J_2, \dots, J_n$ --- пуассоновские случайные величины с показателем $\lambda = pN$.
Положим степень $k$-го клиента равной $\min(J_k, N)$. После того, как все степени клиентов выбраны, 
выбираем нужное количество рёбер равномерным образом. Обозначим $J = \sum\limits_{k=1}^n J_k$. Заметим, что $E(H_p) \leqs J$.

%лемма 2
\begin{lemma} \label{l2}
Возьмём граф $H_p$ при условии того, что $J = m$. 
После этого добавим несколько рёбер, пока в графе не окажется ровно $m$ ребер. 
Очередное ребро будем добавлять так: равномерно выбираем клиента среди клиентов с не максимальной степенью, 
затем равномерно выбираем ещё не проведенное ребро из этого клиента.

Тогда полученный граф по распределению будет совпадать с графом $G_m$.
\end{lemma}
\begin{proof}
Для начала заметим, что мы можем следить только за распределением степеней клиентов, 
поскольку ребра из них во всех случаях проводятся равномерно. В частности когда мы строили $G_m$, 
мы могли сначала выбрать все степени клиентов, и только после этого начать проводить рёбра.

Заметим, что мы знаем, что $J_1, J_2, \dots, J_n$ при условии $J = m$ имеет мультиномиальное распределение 
$Multinom\big(m, \{p_1=p_2= \dots = p_n = \frac{1}{n} \}\big) = \big(Z_1, Z_2, \dots, Z_n\big)$. 
Значит, распределение степеней графа $H_p$ при условии $J = m$ такое же,
как у вектора $Z_{\min} = \big(\min(Z_1, N), \min(Z_2, N), \dots, \min(Z_n, N)\big)$.
Попробуем описать словами этот случайный вектор.

Пусть у нас есть $n$ коробок. Мы берём $m$ шариков, и равномерно кидаем их в коробки. 
После этого во всех коробках, в которых оказалось больше $N$ шариков, лишние выкидываются.
Распределение количеств шариков в коробках как раз будет $Z_{\min}$. 

Вернемся к тому, как мы модифицируем граф $H_p$, а точнее, что происходит со степенями клиентов. 
Мы начинаем с вектора $Z_{\min}$ и начинаем докидывать шарики в коробки, 
равномерно по тем коробкам, в которых меньше $N$ шариков. 
Останавливаемся мы тогда, когда всего шариков во всех коробках будет ровно $m$ штук.

Заметим, что получится то же самое, если бы мы с самого начала загадали число $m$, кидали шарики в коробки равномерно, 
но если в коробке, в которую мы кидаем шарик, уже есть $N$ шариков, то новый шарик пропадает. 
Останавливаемся мы в тот момент, когда в коробках будет ровно $m$ шариков. 
Нетрудно понять, что то же самое происходит и с распределением степеней графа $G_m$.
\end{proof}

\smallskip

\begin{cons} \label{cons_1}
Условие $J = m$ в лемме можно заменить на $J \leqs m$.
\end{cons}
\begin{proof}
Очевидно из доказательства леммы.
\end{proof}

\smallskip

\begin{proof}[Отступление]
Вернёмся к графу $H_p$ и посмотрим, как часто происходит следующее событие: $\{\exists k : J_k \geqs N\}$.
Положим $s(p, N) = \PRob(X \geqs N)$, где $X$ -- пуассоновская случайная величина с показателем $pN$.
Иногда аргументы мы будем опускать и писать просто~$s$.
Заметим, что $\PRob\big(\exists k : J_k \geqs N\big) \leqs N \cdot s(p, N)$. 
Теперь воспользуемся неравенством (\ref{gr_1}) и получим следующие неравенства:
\begin{align}\label{ot_1}
&s(p,N) = \PRob(X \geqs N) \leqs \frac{e^{-pN}(e\cdot pN)^N}{N^N} = (e^{1-p} p )^N,
\\ \label{s_1}
&\PRob(\exists k : J_k \geqs N) \leqs N \cdot s(p, N) = N \cdot (e^{1-p} p )^N.
\end{align} 
Предположим, что $p \in [0, \frac{1}{2}]$, тогда $e^{1-p} p \leqs \frac{\sqrt{e}}{2}$, 
поскольку $(e^{1-p} p)'_p = (1-p) e^{1-p} > 0$. 
Вспомним, что мы предпологали, что $N \geqs 100$. Далее, получаем, что 
\begin{equation}
s \leqs \left(\frac{\sqrt{e}}{2}\right)^N \leqs \frac{e^{50}}{2^{100}} < 10^{-8}.
\end{equation}
Аналогичным образом получаем, что 
\begin{equation}
\PRob\big(\exists k : J_k \geqs N\big) \leqs N \cdot \left(\frac{\sqrt{e}}{2}\right)^N
\leqs 100 \cdot \left(\frac{\sqrt{e}}{2}\right)^{100} < 10^{-6}.
\end{equation}
Тем самым мы поняли, что если $p \in [0, \frac{1}{2}]$, то $s$ и $\PRob(\exists k : J_k \geqs N)$ достаточно маленькие.
\end{proof}

Теперь вернёмся к графу $H_p$ и докажем лемму, аналогичную лемме \ref{l1}.

%лемма 3
\begin{lemma} \label{l3}
Пусть $A$ - множество всех графов, который обладают некоторым монотонным свойством, а $\delta \in (0,1)$.
Тогда, если $\PRob( H_p \in A) \geqs 1 - \eps$, то
\begin{equation} \label{l3_1}
\PRob(G_{\lceil pnN(1+\delta) \rceil} \in A) > 1 - \frac{\eps}{1 - \exp\left(-\frac{\delta^2}{4}pnN\right)},
\end{equation}
a если $\PRob( H_p \in A) \leqs \eps$, то
\begin{equation}\label{l3_2}
\PRob(G_{\lfloor pnN(1-\delta) \rfloor} \in A) < \frac{\eps}{1 - Ns - \exp\left(-\frac{\delta^2}{4}pnN\right)}.
\end{equation}
\end{lemma}

\begin{proof}
Доказательство полностью аналогично доказательству леммы \ref{l1}, надо только применять следствие \ref{cons_1} из леммы \ref{l2}
и доказать следующие неравенства, 
аналогичные неравенствам (\ref{l1_3} - \ref{l1_4}) и (\ref{l1_5}) леммы \ref{l1}:
\begin{align} \label{l3_3}
&\PRob\big( E(H_p) \leqs m \big) > 1 - \exp\left(-\frac{\delta^2}{4}pnN\right),
\\ \label{l3_4}
&\PRob\big( E(H_p) \geqs l \big) > 1 - Ns - \exp\left(-\frac{\delta^2}{4}pnN\right),
\end{align}
где $m = \lceil pnN(1+\delta) \rceil$ a  $l = \lfloor pnN(1-\delta) \rfloor$.

Докажем неравенство (\ref{l3_3}). Вспомним, что $E(H_p) \leqs J$, где $J = \sum\limits_{k = 1}^n J_k$. Заметим, что
\begin{equation}\label{l3_5}\begin{aligned}
&\PRob\big( E(H_p) \leqs m \big) > \PRob\big( J < m \big) \geqs \PRob\big( J < pnN(1+\delta) \big) 
=
\\
= 
&~1 - \PRob\big( J \geqs pnN(1+\delta) \big) \geqs 1 - \frac{e^{-pnN}(epnN)^{pnN(1+\delta)}}{(pnN(1+\delta))^{pnN(1+\delta)}}
=
\\
=
&~1 - \left( \frac{e^{-1} e^{1+\delta}(pnN)^{1+\delta}}{(pnN)^{1+\delta}(1+\delta)^{1+\delta}} \right)^{pnN}
=
1 - \left( \frac{e^\delta}{(1+\delta)^{1+\delta}} \right)^{pnN},
\end{aligned}\end{equation}
где в последнем неравенстве было замечено, что $J$ --- пуассоновская случайная величина с показателем $pnN$, 
и было применено неравенство (\ref{gr_1}).
Теперь заметим, что 
\begin{equation}\begin{aligned}
&\PRob\big( E(H_p) \leqs m \big) > 1 - \left( \frac{e^\delta}{(1+\delta)^{1+\delta}} \right)^{pnN}
=
\\
=
&~1 - \exp\big(pnN \cdot (\delta - (1+\delta) \ln(1+\delta)) \big)
\geqs
 1 - \exp\left(-\frac{\delta^2}{4}pnN\right),
\end{aligned}\end{equation}
где мы применили неравенство $\delta - (1+\delta) \ln(1+\delta) \leqs - \frac{\delta^2}{4}$, 
которое верно для $\delta \in (0,1)$. Тем самым неравенство (\ref{l3_3}) доказано.

Теперь докажем неравенство (\ref{l3_4}). Пусть $A$ --- событие $\{\exists k : J_k \geqs N\}$.
Вспомним, что $\PRob(A) \leqs Ns$, что сказано в неравенстве (\ref{s_1}). Заметим, что
\begin{equation}\begin{aligned}
\PRob\Big( E(H_p) &\geqs l \Big) = 
	\PRob\Big(E(H_p) \geqs l~\big|~A\Big) \cdot \PRob(A) + \PRob\Big(E(H_p) \geqs l~\big|~\overline{A}\Big) \cdot \PRob(\overline{A})
\geqs
\\
\geqs
&~\PRob\Big(J \geqs l~\big|~\overline{A}\Big) \cdot \PRob(\overline{A})
	=
\PRob\Big(J \geqs l~\cap~\overline{A} \Big) 
	\geqs
\PRob\Big(J \geqs l \Big) - \PRob(A),
\end{aligned}\end{equation}
поскольку при условии $\overline{A}$ верно, что $E(H_p) = J$. Продолжим оценивать:
\begin{equation}\begin{aligned}
&\PRob\Big( E(H_p) \geqs l \Big) 
	\geqs 
\PRob\Big(J \geqs l \Big) - \PRob(A) 
	>
\PRob\Big(J > l \Big) - Ns
	\geqs
\\
	\geqs
&~\PRob\Big(J > pnN(1-\delta) \Big) - Ns
	=
1 - Ns - \PRob\Big( J \leqs pnN(1-\delta)  \Big).
\end{aligned}\end{equation}
Теперь, аналогично (\ref{l3_5}), используя неравенство (\ref{gr_2}), получаем:
\begin{equation}\begin{aligned}
\PRob\big( J \leqs&~pnN(1-\delta) \big) 
	\leqs
\frac{
	e^{-pnN}(epnN)^{pnN(1-\delta)}
}{
	(pnN(1-\delta))^{pnN(1-\delta)}
}
	=
\left( \frac{e^{-\delta}}{(1-\delta)^{(1-\delta)}} \right)^{pnN}
	=
\\
	=
&~\exp\big(pnN \cdot (-\delta - (1-\delta)\ln(1-\delta)) \big)
	<
\exp\left(-\frac{\delta^2}{4}pnN\right),
\end{aligned}\end{equation}
где в последнем переходе использовалось неравенство $-\delta - (1-\delta)\ln(1-\delta) < - \frac{\delta^2}{4}$, 
которое верно при $\delta \in (0,1)$. Неравенство (\ref{l3_4}) доказано, а значит и лемма тоже.
\end{proof}

\smallskip

Вернёмся к основной задаче. 
Как и в прошлый раз, мы будем интересоваться
$\PRob(Q(H_p) \geqs qn)$, где $q$ --- некоторое число, меньшее $p$.
Более того, предположим, что $q < p - s$.
Введём случайные величины $T_1, T_2, \dots, T_N$, заданные так: $T_i = 1$ если $i$-й фрагмент скачали хотя бы $qn$ клиентов, 
в противном случае $T_i = 0$. Заметим, что 
\begin{equation}
\PRob\Big(Q(H_p) \geqs qn\Big) = \PRob\left(\sum\limits_{i=1}^N T_i = N\right).
\end{equation}
Посчитаем $\PRob(T_i = 1)$. Ясно, что эта вероятность не зависят от выбора $i$, а значит, 
не умаляя общности, мы можем посчитать только $\PRob(T_1 = 1)$.

Пусть $I_1, I_2, \dots, I_n$ --- бернулевские случайные величины, такие, что $I_i = 1$ тогда и только тогда, 
когда $i$ клиент скачал первый фрагмент. Из построения графа $H_p$ ясно, что $I_i$ не зависимы.
Посчитаем $\PRob\big(I_i = 1\big) = \PRob\big(I_1 = 1\big)$ следующем образом:
\begin{equation}\begin{aligned}
&\PRob\big(I_1 = 1\big) = \sum_{i = 0}^\infty \PRob\big(J_1 = i\big) \cdot \PRob\Big(I_1 = 1 ~\big|~ J_1 = i\Big) 
	= 
\\
	= 
&~\sum_{i = 0}^N e^{-\lambda} \frac{\lambda^k}{k!} \cdot \frac {k}{N} + \PRob\big(J_1 > N\big)
	=
\sum_{i = 0}^{\infty} e^{-\lambda} \frac{\lambda^k}{k!} \cdot \frac {k}{N}  
	- 
	\\
	-
&\sum_{i = N + 1}^{\infty} e^{-\lambda} \frac{\lambda^k}{k!} \cdot \left(\frac {k}{N} - 1\right)
	=
\frac{\lambda}{N} - \frac{\lambda}{N} \cdot s + s - e^{-\lambda} \frac{\lambda^N}{N!},
\end{aligned}\end{equation}
где $\lambda = pN$. Из неравенств выше видно, что $r = \PRob(I_1 = 1) \in (p - s, p)$.

Вернёмся к подсчёту $\PRob(T_1 = 1)$:
\begin{equation}\begin{aligned}
\PRob(T_1 = 1) = \PRob\left(\sum_{i=1}^n I_i \geqs qn\right) 
	= 
1 - \PRob\left(\sum_{i=1}^n I_i < (1-\delta)n\right)
	>
\\
	>
1 - \exp \left( - \frac{rn \delta^2}{2} \right),
\end{aligned}\end{equation}
где $\delta = 1 - \frac{q}{r}$. Мы воспользовались тем, что $I_i$ --- независимы, и в очередной раз применили теорему Чернова. 
Заметим, что $p - s < r$, кроме того, мы предположили, что $q < p-s$, а значит, $q < r$, т.е. $\delta > 0$.

Пусть $T = \sum\limits_{i = 0}^N T_i$. Мы только что доказали, что
\begin{equation}
\Expect T > N \cdot \left( 1 - \exp \left( - \frac{rn \delta^2}{2} \right) \right).
\end{equation}
Заметим, что 
\begin{equation}
\Expect T \leqs N \cdot \PRob\big(T = N\big) + (N-1) \cdot \Big(1 - \PRob\big(T = N\big)\Big) = N + \PRob\big(T = N\big) - 1,
\end{equation}
откуда получим следующее:
\begin{equation}\begin{aligned}
\PRob(T = N) \geqs \Expect T + 1 - N > N \cdot \left(1 - \exp \left( - \frac{rn \delta^2}{2} \right)\right)  + 1 - N
=
	\\
=
1 - N \cdot \exp \left( - \frac{rn \delta^2}{2} \right).
\end{aligned}\end{equation}
Запишем итоговое неравенство:
\begin{equation}\label{t3_1} \begin{aligned}
\PRob\Big(Q(H_p) \geqs qn \Big) 
	> 
1 - N \cdot \exp\left(- \frac{n (r-q)^2}{2r} \right)
	>
\\ 
	>  
1 - N \cdot \exp\left(- \frac{n (p - q - s)^2}{2p} \right). 
\end{aligned}\end{equation}

\smallskip

Теперь всё готово для теоремы, аналогичной теореме \ref{t1}.

%теорема 3
\begin{theorem}\label{t3}
Пусть $q, \sigma \in (0, 1)$. Возьмем 
\begin{align}
& m = \min(nN, \lceil pnN + 2\sqrt{pnN} \rceil), ~ \text{ где $p$ такое, что} 
	\\
& p \geqs q + s(p, N) + c + \sqrt{c^2+2(q+s(p, N))c}, \text{ a} \label{t3_2}
	\\
& c = \frac{\ln(2N) - \ln(\sigma)}{n}.
\end{align}
Тогда в графе $G_m$ с вероятностью, большей, чем $1 - \sigma$, все фрагменты будут скачены хотя бы $qn$ клиентами:
\begin{equation}\label{t3_3}
\PRob\big(Q(G_m) \geqs qn\big) > 1 - \sigma.
\end{equation}
\end{theorem}

\begin{proof}
Доказательство дословно переносится с доказательства теоремы \ref{t1}, 
где в качестве неравенства (\ref{t1_t3_1}) выступает неравенство (\ref{t3_1}). 
Также, вместо леммы \ref{l1} и неравенства (\ref{l1_1}) выступает лемма \ref{l3} и неравенство (\ref{l3_1}).
\end{proof}

В только что доказанной теоремы есть небольшой изъян -- на $p$ наложено достаточно сложное условие. 
Постараемся его немного упростить.
\begin{cons}
Пусть $(q+c) < \frac{1}{4}$, тогда условие на $p$ можно заменить на следующее:
\begin{equation}
p \geqs 2\left(q + c + \exp\left(-\frac{N}{18}\right) \right)
\end{equation}

\begin{proof}
Для начала докажем, что для $p = 2\left(q + c + \exp\left(-\frac{N}{18}\right) \right)$ выполняется условие (\ref{t3_2}).
Заметим, что $p < \frac{1}{2} + \frac{2}{e^5} < \frac{2}{3}$, поскольку $N \geqs 100$.
Также вспомним неравенство (\ref{ot_1}) и улучшим его:
\begin{equation}
s(p, N) \leqs (e^{1-p} p)^N 
	=
\exp\Big(N\cdot \big( \ln(p) + (1-p) \big)\Big)
	\leqs
\exp\left( - N \cdot \frac{(1-p)^2}{2}\right),
\end{equation}
где мы применили неравенство $\ln(p) + (1-p) \leqs - \frac{(1-p)^2}{2}$, которое верно для $p\in(0,1)$.
Теперь заметим, что, поскольку $p < \frac{2}{3}$, то 
\begin{equation}
s(p, N) \leqs \exp\left( - N \cdot \frac{(1-p)^2}{2}\right) < \exp\left( - N\cdot \frac{1}{18}\right),
\end{equation}
откуда мы получаем, что 
\begin{equation}
p > 2(q+c+s(p,N)) \geqs q + s(p, N) + c + \sqrt{c^2+2(q+s(p, N))c}.
\end{equation}

Итак, мы поняли, что для $p = 2\left(q + c + \exp\left(-\frac{N}{18}\right) \right)$ выполняются условия теоремы, 
а значит верно неравенство (\ref{t3_3}). 
Заметим, что с увеличением $p$ увеличивается и $m$, а значит, и $\PRob\big(Q(G_m) \geqs qn\big)$.
Таким образом, для $p > 2\left(q + c + \exp\left(-\frac{N}{18}\right) \right)$ так же будет выполнено неравенство (\ref{t3_3}).
\end{proof}
\end{cons}


%модель 3
\newpage
\section*{Модель 3. Выбор редкого фрагмента.}

В этой модели $G_m$ будет строиться так: сначала выбираем случайного клиента, а затем выбираем наиболее редкий фрагмент и скачиваем его.
Если же таковых несколько, то можно либо выбирать любой из них, либо с минимальным индексом. 
На дальнейшие рассуждения это влиять не будет.

%лемма 4
\begin{lemma}\label{l4}
Пусть $m \leqs \frac{nN}{5}$ и $\delta < 2e - 1$, тогда вероятность того, что в графе $G_m$ существует клиент, 
со степенью большей чем $(1+\delta)\frac{N}{5}$ меньше, чем $n \exp\left(- \frac{\delta^2 N}{20}\right)$.
\end{lemma}
\begin{proof}
Заметим, что от того, что мы возьмём $m$ побольше, эта вероятность только увеличится. 
А значат, мы имеем право рассуждать только про $m = \frac{nN}{5}$. 
Пусть $T_1, T_2, \dots, T_m$ -- случайные величины, такие, что $T_i = 1$ тогда и только тогда, 
когда мы на $i$-том шагу выбрали $1$-го клиента.
Заметим, что 
\begin{equation}
\PRob\left(\sum_{i = 1}^m T_i > (1+\delta)\frac{m}{n} \right) < \exp\left(- \frac{\delta^2 N}{20} \right),
\end{equation}
поскольку $T_i$ -- независимые бернуллиевские случайные величины с показателем $\frac{1}{n}$, и мы применили к ним теорему Чернова.
Тем самым мы оценили вероятность того, что первый клиент имеет стемень большую, чем $(1+\delta)\frac{nN}{5}$.
Ясно, что искомая вероятность не более чем в $n$ раз больше.
\end{proof}

Итак, мы по прежнему будем следить за функцией $Q(G_m)$. Для этого введём ещё одну функцию $W(G_m)$, 
которая будет выдавать максимальную степень фрагмента.
Кроме того, введем $R(m) = W(G_m) - Q(G_m)$. 
Ясно, что $R(m+1) - R(m)$ может быть только $\{-1,0,1\}$, поскольку мы добавили всего одно ребро в граф, 
а значит изменили степень только одного фрагмента и то на $1$. 

\begin{theorem}
Пусть $m \leqs \frac{nN}{5}$, тогда
\begin{equation}
\PRob\left(\exists l \leqs m : R(l) > 2 \right) < n\exp\left(-\frac{N}{5}\right) +  \exp\left(- \frac{N}{20}\right)
\end{equation}

\begin{proof}
Разыграем растущий граф: $G_1, G_2, \dots G_m$. Пусть для этого растущего графа нашлось такое $l \leqs m$, что $ R(l) > 2$.
Обозначим за $m_2$ минимальное такое $l$. Заметим, что $R(m_2 - 1) = 2$. 
Теперь обозначим за $m_1$ максимальное $l$ такое, что $l < m_2$ и $R(l-1) = 1$.
Таким образом, $R(m_2) = 3, R(m_1 - 1) = 1$, и $R(l) = 2$ для любого $l \in [m1, m2)$.

Пусть $D = 3 \cdot \frac{N}{5}$. Пусть степени всех клиентов графа $G_m$ не более, чем $D$. 
Заметим, что вероятность этого хотя бы $1 -  n\exp\left(-\frac{N 2^2}{20}\right)$ по лемме \ref{l4}.
Также заметим, что степени всех клиентов графов $G_1, G_2, \dots G_m$ не превосходят $D$.

Давайте ещё немного поисследуем ситуацию, которую мы имеем. Например мы знаем, что в момент $m_1$ есть 
только один фрагмент с максимальной степенью. Обозначим эту степень за $L + 1$, тогда минимальная степень будет $L - 1$.
Также заметим, что до момента $m_2$ других степеней не будет, т.к. минимум и максимум не могли меняться, т.к. иначе бы 
наблюдалось изменение $R(l)$. 

Заметим, что в момент $m_1 - 1$ 
количество фрагментов степени $L$ было хотя бы $N-D$, т.к. чтобы сотворить фрагмент степени $L+1$ необходимо, 
чтобы все нескаченные фрагменты выбранным клиентом были бы степени $L$ или большей (но больших степеней точно нет), а вот минимальная антистепень 
клиента точно хотя бы $N - D$. Аналогичные рассуждения верны для момента $m_2 - 1$, т.е. в этот момент должно быть хотя бы $N - D$
вершин степени $L+1$. 

Я утверждаю, что в момент $\min(m_1 - 1 + N, m_2 - 1)$ будет хотя бы $N-D$ фрагментов степени $L+1$.
Пусть $m_1 - 1 + N < m_2 - 1$, иначе мы это уже выяснили.
Заметим, что в момент $m_2 - 1$ фрагментов степени $L$ хотя бы $N-D$, а значит,
фрагментов степени $L - 1$ не более $D$. Далее, за каждый шаг $(m_1 - 1) \to (m_1)$, $(m_1) \to (m_1 + 1)$, \dots, 
$(m_1 - 2 + N) \to (m_1 - 1 + N)$
 мы либо увеличиваем количество фрагментов степени $L+1$, либо уменьшаем количество фрагментов степени $L-1$.
Поскольку фрагменты степени $L-1$ не пропали, значит мы как минимум $N-D + 1$ раз увеличили количество фрагментов $L+1$, 
т.е. в момент $m_1 - 1 + N$ таких фрагментов будет больше, чем $N - D$.

Разберёмся теперь, каким образом увеличивается количество фрагментов степени $L+1$. 
Ясно, что чтобы такое произошло, необходимо выбрать клиента, который бы скачал все фрагменты степени $L-1$. 
Не вызывает сомнений, что таких клиентов не может быть больше, чем $L-1$, поскольку они все скачали какой-то фрагмент степени $L-1$.
Итак, давайте на каждом шагу нумеровать клиентов так: первые номера получат те клиенты, 
которые скачали все фрагменты степени $L-1$, остальные клиенты получают последующие номера. 
Заметим, что для того, чтобы за $\min(N, m_2 - m_1)$ шагов как минимум $N-D$ раз увеличить количество фрагментов степени $L+1$,
необходимо как минимум $N-D$ раз выбрать клиента с номером не более, чем $L-1$.

Тем самым, вероятность того, что в момент $\min(m_1 - 1 + N, m_2 - 1)$ будет хотя бы $N-D$ фрагментов степени $L+1$, 
не более, чем $ A = \PRob(I_1 + I_2 + \dots + I_{\min(N, m_2 - m_1)} > N - D)$, 
где $I_k$ -- независимые бернуллиевские случайные величины с показателем $\frac{L-1}{n}$.

Вспомним, что всё это происходило при допущении, что степени всех клиентов не более $D$, 
а значит, $\PRob\left(\exists l \leqs m : R(l) > 2 \right) < A +  n\exp\left(-\frac{N}{5}\right)$, 
поскольку либо должен будет найтись клиент с большой степенью, 
либо к моменту $\min(m_1 - 1 + N, m_2 - 1)$ должно будет найтись много фрагментов степени $L+1$.

Осталось доказать, что 
\begin{equation}
A < \exp\left(- \frac{n}{8}\right).
\end{equation}
Заметим, что $(L-1) N < m_1 < \frac{nN}{5}$, т.к. $L-1$ -- минимальная степень фрагмента в момент $m_1$.
Значит, $\frac{L-1}{n} < \frac{1}{5}$. Заметим, что
\begin{equation}
A \leqs \PRob\left(\sum_{k = 1}^N I_k > N - D\right) \leqs \PRob\left(\sum_{k = 1}^N J_k > N - D\right),
\end{equation}
где $J_k$ -- независимые бернуллиевские случайные величины с показателем $\frac{1}{5}$. 
Заметим, что $N - D = 2 \frac{N}{5}$, а значит мы можем применить 
теорему Чернова для $\sum\limits_{k = 1}^N J_k$ и получить
\begin{equation}
\PRob\left(\sum_{k = 1}^N J_k > (1 + 1) \frac{N}{5}\right) < \exp\left( - \frac{N}{20}\right)
\end{equation}

\end{proof}

\end{theorem}

%\section*{Экспериментальные данные}


	%=======================================================================================
\newpage

\begin{thebibliography}{XXXX}
	{% Открывающая фигурная скобка небходима
			\renewcommand{\baselinestretch}{1.01}
			\selectfont
			
		\bibitem{cool}{Liao W.-G., Papadopoulos F., Psounis K., Psomas C.}	
		{\em Modeling BitTorrent-like systems with many classes of users.}
		// ACM Trans. Model. Comput. Simul. 23(2): 13 (2013)
			
		\bibitem{unavalib}{Kaune S., Cuevas R., Tyson G., Mauthe A., Guerrero C., Steinmetz R.} 
		{\em Unraveling bittorrent's file unavailability: Measurements, analysis and solution exploration.}
		// arXiv:0912.0625v1 (2009)
				
		\bibitem{price}{Price, D. J. de S.}
		{\em A general theory of bibliometric and other cumulative advantage processes.} 
		// J. Amer. Soc. In-form. Sci. 27, 292–306 (1976)
	
		\bibitem{newman}{Mark E. J. Newman.}
		{\em The Structure and Function of Complex Networks.}
		// SIAM Review 45(2): 167-256 (2003)

		\bibitem{spec}{}
		{\em Bittorrent Protocol Specification v1.0} //
		\\ URL: https://wiki.theory.org/BitTorrentSpecification
		
		\bibitem{chernov}{John Canny}
		{\em Lecture 10, Chernoff Bounds.}
		// CS174, University of California, Berkeley, (2001)
		
		\bibitem{chernov_gr}{Mitzenmacher M., Upfal E.}
		{\em Probability and Computing: Randomized Algorithms and Probabilistic Analysis.}
		// p. 97, Cambridge University Press.  (2005)
		
	}% Закрывающая скобка, парная к открывающей
\end{thebibliography}
	

\end{document}
